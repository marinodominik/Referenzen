\documentclass{scrartcl}
\usepackage{selinput}
\usepackage[ngerman,english]{babel} % Standard: English
\usepackage{graphicx}
\usepackage{varioref}

\usepackage{color}
\definecolor{lightgray}{gray}{0.95}




%%%%%%%%%%%%%%%%%%%%%%%%%%%%%%
% For including Java-sources %
%%%%%%%%%%%%%%%%%%%%%%%%%%%%%%

\usepackage{listings}
\lstset{
   language=Java,
   backgroundcolor=\color{lightgray},
   extendedchars=true,
   basicstyle=\footnotesize\ttfamily,
   showstringspaces=false,
   showspaces=false,
   frame=leftline,
   numbers=left,
   numberstyle=\tiny, % or e.g. \footnotesize
   numbersep=9pt,
   tabsize=2,
   breaklines=true,
   showtabs=false,
   captionpos=b,
}




%%%%%%%%%%%%%%%%%%%%%%%%%%%%%%
% Definitions for signatures %
%%%%%%%%%%%%%%%%%%%%%%%%%%%%%%
\newcommand*{\TwoSignatures}[2]{%
\vspace{1.2cm}
\noindent\begin{tabular}{ll}%
\makebox[6.8cm]{\hrulefill} & \makebox[6.8cm]{\hrulefill}\\
#1 & #2
\end{tabular}

}
\newcommand*{\ThreeSignatures}[3]{%
\vspace{1.2cm}
\noindent\begin{tabular}{lll}%
\makebox[4.4cm]{\hrulefill} & \makebox[4.4cm]{\hrulefill} & \makebox[4.4cm]{\hrulefill}\\
#1 & #2 & #3
\end{tabular}
}




%%%%%%%%%%%%%%%%%%%%%%%%%%%%%%
% For cites and bibliography %
%%%%%%%%%%%%%%%%%%%%%%%%%%%%%%
\usepackage[babel,german=guillemets]{csquotes}
\usepackage[backend=biber]{biblatex}
\addbibresource{bibliography.bib}


\usepackage{graphicx} %package to manage images
\graphicspath{ {images/} }


\begin{document}
\subject{Weekly Report of OOP-Project SS17---Biemann}
\title{Report 1}% Report 1, Report 2 and so on
\subtitle{Group 12 (Medical Hospice Application)}
\author{Marino, Dominik\\Stergiou, Charalambos\\ Hamiani, Samir}%Use format lastname, surname in separated lines for each participant
\date{\today}
\maketitle
%\tableofcontents % Use this only if you have more than two pages

%%%%%%%%%% Let's start %%%%%%%%%%

% The following is an example structure you can use. Feel free to adapt this to your needs.
% Remark: Sectioning makes only sense if you have enough content/text. Avoid sections with less than a few lines of text.
\section{Motivation}
%In this section you should write about the tasks and your job definitions.
%You can talk about some basic stuff, but you can also use an own section for this.
In this week we thought about what should in it, what is important and what can  the application.  Our idea is that the user can see every important information at the beginning. If you click on the table, the user (in every case we mean doctor and nurses) can change and update the patient.

\section{Realization}
\subsection{Work of Dominik Marino}
In this week I create a (beta version) graphical user interface (GUI) for the application.I created it with the Scene Builder 2.0 for a first look up that we can easier image the project.Fist of all the user should login and then the program decide what kind of job you have. In the doctor interface you can see a table that all important information about the patients. The last one is the data of the patients.

%Maybe you want to split to separate the details of each work.
%\subsection{Work of John Doe}
\subsection{Work of Samir Hamiani}
This week I was thinking about our project and how we can realize it. By using a class state diagram I became a overview of the project. I worked with my team member on the project application sheet. I have also worked with my team on the functional requirements for our project and made a agreement about our weekly meeting for talking about our project and eventual issues.
%Do not forget any ressources you have used and cite them like in %\cite{example:url}.
%
\subsection{Work of Charalambos Stergiou}
The goal for this week, was to identify the different use cases for the acting People. The Doctor and the Nurse. In this week i was looking for the use case of the Nurse. 
\begin{itemize}
\item The Nurse
\begin{itemize}
\item Should have an overview of the preliminary work of the other nurse. Did she reach her goal.
\item The goal is that a patient had drink enough water in a period of time. The status of that should be display in a traffic light.
\item The nurse have the ability to document if the patient had drink the served amount of water.
\item The nurse should see notification or orders from the doctor.
 
\end{itemize}

\end{itemize}


\section{Conclusion}
%Use this for your weekly conclusion and if applicable the outlook on %next week.
In the next few day we creating the basic structure like ER-model for the database and UML-diagram and the last one the use-case-diagram.
\printbibliography

% Do not forget your signatures
%\TwoSignatures{John Doe}{Martin Mustermann} % For four signatures use it twice
\ThreeSignatures{Dominik Marino}{Charalambos Stergiou}{Samir Hamiani}

\end{document}